\documentclass[a4paper,12pt]{article}

% Packages
\usepackage{graphicx}  % For images
\usepackage{amsmath}   % For math symbols
\usepackage{hyperref}  % For hyperlinks
\usepackage{fancyhdr}  % For headers
\usepackage{geometry}  % For page layout
\usepackage{xcolor}    % For colors
\usepackage{listings}  % For code snippets
\usepackage{float}     % For image positioning
\usepackage{caption}
\usepackage[utf8]{inputenc} % For special characters
\geometry{a4paper, margin=1in}

% Fancy header/footer settings
\pagestyle{fancy}
\fancyfoot[R]{\thepage} % Right-align page number in the footer

% Title information
\title{Zephyr RTOS Project Report}
\author{
107474-Joseane Pereira \\
109050-Gabriel Costa \\
Universidade de Aveiro, DETI
}
\date{\today}

\begin{document}
\begin{figure}
    \centering
    \includegraphics[width=0.3\linewidth]{ua.pdf}
    \label{fig:enter-label}
\end{figure}
\maketitle
\newpage
\tableofcontents
\newpage

\section{Introduction}
The aim of this project is to apply the Linux Real-Time Services and the Real-Time Model to the development of a real-life inspired real-time application. The project encompasses a set of cooperating tasks, involving synchronization, shared resources, access to a real-time database, etc.

\section{Architecture}
\subsection{System Overview}
In this project, the real-time monitoring system is structured around several primary tasks, each with distinct roles that simulate components in an industrial environment:
\begin{itemize}
    \item \textbf{Task 0}: Responsible for updating the RTDB with the button states.
    \item \textbf{Task 1}: Updates the LED states based on the button states from the RTDB.
    \item \textbf{Task 2}: Validates RTDB entries and resets them if they are corrupted.
    \item \textbf{Task 3}: Placeholder for future tasks or additional functionality.
\end{itemize}

\subsection{Data Structures and Access Methods}
The primary data structure used in this project is the Real-Time Database (RTDB). The RTDB is used to store the states of LEDs and buttons, ensuring synchronized access and updates by different tasks.

\subsection{Synchronization Mechanisms}
The implementation relies on multiple threads that perform specific tasks concurrently. To ensure proper coordination, various synchronization methods are applied:
\begin{itemize}
    \item \textbf{Thread Suspension and Resumption}: Tasks suspend themselves after execution and are resumed by the scheduler.
    \item \textbf{Mutexes}: Used to control access to shared resources and prevent data corruption.
\end{itemize}

\section{Task Priorities and Activation Rates}
Each task in the system is assigned a specific priority and activation rate to ensure timely execution and prevent task starvation. The following table outlines the task priorities and activation periods.

\begin{table}[H]
    \centering
    \begin{tabular}{|l|c|c|p{5cm}|}
        \hline
        \textbf{Task} & \textbf{Priority} & \textbf{Activation Period} & \textbf{Description} \\
        \hline
        Task 0 & 1 & 50ms & Updates the RTDB with the button states. \\
        \hline
        Task 1 & 2 & 100ms & Updates the LED states based on the button states from the RTDB. \\
        \hline
        Task 2 & 1 & 100ms & Validates RTDB entries and resets them if they are corrupted. \\
        \hline
        Task 3 & 5 & 150ms & Placeholder for future tasks or additional functionality. \\
        \hline
    \end{tabular}
    \caption{Task Priorities and Activation Periods}
    \label{tab:task_priorities}
\end{table}

\section{System Schedulability}
To confirm that all tasks meet their deadlines, we perform a schedulability analysis using the system's utilization factor.

\textbf{Utilization Factor (U)}: Given that each task is independent and periodic, we can calculate the utilization factor \( U \) using:

\[
U = \sum_{i=1}^n \frac{C_i}{T_i}
\]

where \( C_i \) is the computation time and \( T_i \) is the period of each task. Assuming each task completes within its assigned period, this calculation helps ensure that the system remains schedulable.

\section{Task Execution Sequence and Relevant Events}
The system follows a well-defined sequence to maintain real-time performance. Below is the order of task execution and relevant interactions:

\begin{enumerate}
    \item \textbf{Task 0}: Updates the RTDB with the button states.
    \item \textbf{Task 1}: Updates the LED states based on the button states from the RTDB.
    \item \textbf{Task 2}: Validates RTDB entries and resets them if they are corrupted.
    \item \textbf{Task 3}: Placeholder for future tasks or additional functionality.
\end{enumerate}

\subsection{Relevant Events and Inter-Task Communication}
\textbf{RTDB Operations}: The RTDB is critical for data synchronization:
\begin{itemize}
    \item \textbf{Task 0} writes to the RTDB, while \textbf{Task 1} and \textbf{Task 2} read from it.
    \item Mutexes manage concurrent access to prevent data corruption.
\end{itemize}

\section{Tests}
The tests for the scheduler verify that tasks are added correctly and meet their deadlines. The test suite includes:
\begin{itemize}
    \item \textbf{Scheduler Initialization Test}: Verifies that the scheduler initializes correctly.
    \item \textbf{Task Addition Test}: Verifies that tasks can be added and appear in the task table.
    \item \textbf{Task Deadline Test}: Verifies that tasks meet their deadlines.
\end{itemize}

\section{Results}
The tests confirm that the scheduler initializes correctly, tasks are added as expected, and all tasks meet their deadlines within the specified margins.

\section{Conclusion}
In this project, a multi-threaded system was successfully implemented to manage tasks in a real-time environment. The scheduler ensures that tasks are executed periodically and meet their deadlines. The RTDB provides synchronized access to shared data, preventing data corruption. Overall, the project demonstrates effective use of periodic task scheduling, synchronization, and real-time data management.

\end{document}